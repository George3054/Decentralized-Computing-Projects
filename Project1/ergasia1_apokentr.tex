\documentclass[10pt]{report}
\usepackage[utf8]{inputenc}
\usepackage[greek]{babel}
\usepackage{listings}
\newcommand{\en}{\selectlanguage{english}}
\newcommand{\gr}{\selectlanguage{greek}}
\usepackage{fancyhdr}
\pagestyle{plain}
\fancyhf{}
\fancyhead[R]{\leftmark \gr ΛΕΚΚΑΣ ΓΕΩΡΓΙΟΣ ΑΜ:1067430}
\fancyhead[LO]{\rightmark }
\fancypagestyle{plain}{}
\usepackage{graphicx}

\title{\textbf{1o  {\en PROJECT} ΑΠΟΚΕΝΤΡΩΜΕΝΟΣ ΥΠΟΛΟΓΙΣΜΟΣ \& ΜΟΝΤΕΛΟΠΟΙΗΣΗ}}
\author{\gr ΛΕΚΚΑΣ ΓΕΩΡΓΙΟΣ ΑΜ:1067430  Έτος 5ο\newline}
\date{}
\cfoot{\thepage}
\begin{document}
	\maketitle	
	\thispagestyle{fancy}
	\tableofcontents
	\pagestyle{plain}
	\begin{itemize}
		\item[A.] \gr Άσκηση 1 
		\item[Β.] \gr Άσκηση 2
		
	\end{itemize}
	\pagebreak
	
	\section*{\gr Α. Άσκηση 1}
	
	\gr Γνωρίζουμε πως σε ένα {\en Randomized} αλγόριθμο υπάρχουν τέσσερις πιθανές ιδιότητες που μπορεί να έχει ένας κόμβος.Μπορεί 1) να είναι {\en Running}, δηλαδή να μην έχει σταθεροποιηθεί το χρώμα του, 2) να είναι {\en Stopped}, δηλαδή να έχει κάποιο χρώμα, 3) να είναι ενεργός, δηλαδή αυτό το γύρο θα προσπαθήσει να βρει χρώμα και 4) να είναι ανενεργός.
	
	\par \gr Στο πρώτο ερώτημα της άσκησης επιθυμούμε ο κόμβος να ενεργοποιείται με πιθανότητα {\en p{\scriptsize a}}. Συνεπώς η ανάλυση η ανάλυση για αυτόν τον αλγόριθμο όταν η πιθανότητα ενεργοποίησης είναι {\en p{\scriptsize a}} είναι η ακόλουθη:\\ \\
	Έστω ότι έχουμε ένα κόμβο {\en u} ο οποίος είναι {\en running} και έχει ως γείτονες {\en k running} κόμβους {\en v}.\\ Τώρα ας υποθέσουμε πως ο {\en u} είναι ενεργός και έχει τουλάχιστον {\en k}+1 διαθέσιμα χρώματα. Τότε {\en A{\scriptsize ri}} είναι η πιθανότητα ο {\en u} να συγκρουστεί με τον {\en v{\scriptsize i}}: 1/{\en k}+1. Δηλαδή {\en P{\scriptsize r}}({\en A{\scriptsize ri}}) $\leq$1/{\en k}+1 (Φράζουμε διότι μπορεί να έχω παραπάνω χρώματα).Άμα ο {\en vi} είναι ανενεργός τότε δεν υπάρχει σύγκρουση.\\Συνεπώς ισχύει το παρακάτω :{\en P{\scriptsize r}}({\en A{\scriptsize ri}}/{\scriptsize {\en vi} ενεργός}) $\leq$1/{\en k}+1.
	\\ \\
	Άμα {\en v{\scriptsize i}} ενεργός με πιθανότητα {\en p{\scriptsize a}} τότε: {\en P{\scriptsize r}}({\en A{{\en v{\scriptsize i}}}) $\leq$p{\scriptsize a}/{\en k}+1.\gr Γενικά θέλω να υπολογίσω : {\en P{\scriptsize r}}({\en A{{\en v{\scriptsize i}}}})
	\par \gr
\end{document}