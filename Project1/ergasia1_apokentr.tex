\documentclass[10pt]{report}
\usepackage[utf8]{inputenc}
\usepackage[greek]{babel}
\usepackage{listings}
\newcommand{\en}{\selectlanguage{english}}
\newcommand{\gr}{\selectlanguage{greek}}
\usepackage{fancyhdr}
\pagestyle{plain}
\fancyhf{}
\fancyhead[R]{\leftmark \gr ΛΕΚΚΑΣ ΓΕΩΡΓΙΟΣ ΑΜ:1067430}
\fancyhead[LO]{\rightmark }
\fancypagestyle{plain}{}
\usepackage{graphicx}
\usepackage{amsthm}

\title{\textbf{1o  {\en PROJECT} ΑΠΟΚΕΝΤΡΩΜΕΝΟΣ ΥΠΟΛΟΓΙΣΜΟΣ \& ΜΟΝΤΕΛΟΠΟΙΗΣΗ}}
\author{\gr ΛΕΚΚΑΣ ΓΕΩΡΓΙΟΣ ΑΜ:1067430  Έτος 5ο\newline}
\date{}
\cfoot{\thepage}
\begin{document}
	\maketitle	
	\thispagestyle{fancy}
	\tableofcontents
	\pagestyle{plain}
	\begin{itemize}
		\item[A.] \gr Άσκηση 1 
		\item[Β.] \gr Άσκηση 2
		
	\end{itemize}
	\pagebreak
	\gr Η συγγραφή της αναφοράς πραγματοποιήθηκε με {\en Latex} με τη βοήθεια του {\en TexStudio}. 
	\section*{\gr \textbf{Α. Άσκηση 1}}
	\subsection*{\gr Ερώτημα 1}
	\gr Γνωρίζουμε πως σε ένα {\en Randomized} αλγόριθμο υπάρχουν τέσσερις πιθανές ιδιότητες που μπορεί να έχει ένας κόμβος.Μπορεί 1) να είναι {\en Running}, δηλαδή να μην έχει σταθεροποιηθεί το χρώμα του, 2) να είναι {\en Stopped}, δηλαδή να έχει κάποιο χρώμα, 3) να είναι ενεργός, δηλαδή αυτό το γύρο θα προσπαθήσει να βρει χρώμα και 4) να είναι ανενεργός.
	
	
	\par \gr Στο πρώτο ερώτημα της άσκησης επιθυμούμε ο κόμβος να ενεργοποιείται με πιθανότητα {\en p{\scriptsize a}}. Συνεπώς η ανάλυση η ανάλυση για αυτόν τον αλγόριθμο όταν η πιθανότητα ενεργοποίησης είναι {\en p{\scriptsize a}} είναι η ακόλουθη:\\
	\par Έστω ότι έχουμε ένα κόμβο {\en u} ο οποίος είναι {\en running} και έχει ως γείτονες {\en k running} κόμβους {\en v}.\\ Τώρα ας υποθέσουμε πως ο {\en u} είναι ενεργός και έχει τουλάχιστον {\en k}+1 διαθέσιμα χρώματα. Τότε {\en A{\scriptsize ri}} είναι η πιθανότητα ο {\en u} να συγκρουστεί με τον {\en v{\scriptsize i}}: 1/{\en k}+1. Δηλαδή {\en P{\scriptsize r}}({\en A{\scriptsize ri}}) $\leq$1/{\en k}+1 (Φράζουμε διότι μπορεί να έχω παραπάνω χρώματα).Άμα ο {\en vi} είναι ανενεργός τότε δεν υπάρχει σύγκρουση.\\Συνεπώς ισχύει το παρακάτω :{\en P{\scriptsize r}}({\en A{\scriptsize ri}}/{\scriptsize {\en vi} ενεργός}) $\leq$1/{\en k}+1.\\
	\par Άμα {\en v{\scriptsize i}} ενεργός με πιθανότητα {\en p{\scriptsize a}} τότε: {\en P{\scriptsize r}}({\en A{{\en v{\scriptsize i}}}) $\leq$p{\scriptsize a}/{\en k}+1.\gr Γενικά θέλω να υπολογίσω : {\en P{\scriptsize r}}({\en A{{\en v{\scriptsize 1}}}}$\cup${\en A{{\en v{\scriptsize 2}}}}$\cup${\en A{{\en v{\scriptsize 3}}}}$\cup$...$\cup${\en A{{\en v{\scriptsize k}}}}). Οπότε εφαρμόζω {\en Union Bound} και έχω: {\en P{\scriptsize r}}({\en A{{\en v{\scriptsize 1}}}}$\cup${\en A{{\en v{\scriptsize 2}}}}$\cup${\en A{{\en v{\scriptsize 3}}}}$\cup$...$\cup${\en A{{\en v{\scriptsize k}}}})$\leq$$\sum_{n=1}^{10}${\en P{\scriptsize r}}({\en A{{\en v{\scriptsize i}}}))$\leq${\en k$\cdot$p{\scriptsize a}}/{\en k}+1$\leq${\en k$\cdot$p{\scriptsize a}}/{\en k}$\leq$p{\scriptsize a}}.\\
	Συνεχίζοντας η πιθανότητα να μην συγκρουστούν οι δύο κόμβοι είναι:\\ {\en P{\scriptsize r}}(1/{\en A{{\en v{\scriptsize 1}}}}$\cup${\en A{{\en v{\scriptsize 2}}}}$\cup${\en A{{\en v{\scriptsize 3}}}}$\cup$...$\cup${\en A{{\en v{\scriptsize k}}}}) = 1-{\en P{\scriptsize r}}({\en A{{\en v{\scriptsize 1}}}}$\cup${\en A{{\en v{\scriptsize 2}}}}$\cup${\en A{{\en v{\scriptsize 3}}}}$\cup$...$\cup${\en A{{\en v{\scriptsize k}}}})$\geq${\en p{\scriptsize a}}.\\
	Άρα {\en P{\scriptsize r}}({\en u} να μην συγκρουστεί και να σταματήσει και να πάρει χρώμα)$\geq${\en p{\scriptsize a}}$\cdot${\en p{\scriptsize a}}$\geq${\en p{\scriptsize a}}$^2$.\\
	\par Βασιζόμενοι στο {\en Lemma} της διαφάνειας 61 του δευτερου σετ διαφανειών έχουμε: {\en P{\scriptsize r}}(να μην σταματήσει)$\leq$1-{\en p{\scriptsize a}}$^2$. Συνεπώς η πιθανότητα να μην σταματήσει σε Τ γύρους είναι: (1-{\en p{\scriptsize a}}$^2$)$\cdot$(1-{\en p{\scriptsize a}}$^2$)$\cdot$...(Τ φορές)$\leq$(1-{\en p{\scriptsize a}}$^2$)$^{\en T}$.
	\par Τέλος για να καταλήξω στο δεύτερο {\en Corollary} της διαφάνειας 61 του δευτερου σετ διαφανειών πρέπει αρχικα να καθορίσω το Τ=Ο($\log_{}{\en n}$).Πρέπει δηλαδή να ορίσω τις σταθερές που κρύβει το Ο.Έχουμε: Τ={\en c}$\cdot$$\log_{\frac{1}{1-{\en p{\scriptsize a}}^2}}{\en n}$. Οπότε ένας κόμβος θα σταματήσει με πιθανότητα $\leq(\frac{1}{\frac{1}{1-{\en p{\scriptsize a}}^2}})^{{\en c}\cdot\log_{\frac{1}{1-{\en p{\scriptsize a}}^2}}{\en n}}\leq(\frac{1}{1-{\en p{\scriptsize a}}^2})^{-1\cdot{\en c}\cdot\log_{\frac{1}{1-{\en p{\scriptsize a}}^2}}{\en n}}\leq{\en n}^{\en -c}$.\\
	Aπό {\en Union Bound} έχουμε : {\en P{\scriptsize r}}(να μην έχει τελίωσει κάποιος κόμβος μετά από Τ γύρους)$\leq{\en n}\cdot{\en n}^{\en -c}={\en n}^{\en -c+1}$. Οπότε η πιθανότητα να μην έχει τερματίσει κάποιος κόμβος μετά από {\en c}$\cdot$$\log_{\frac{1}{1-{\en p{\scriptsize a}}^2}}{\en n}$ γύρους είναι το πολύ $\frac{1}{{\en n}^{\en c-1}}$. Συμπερασματικά, η πιθανότητα να τερματίσει μετά από {\en c}$\cdot$$\log_{\frac{1}{1-{\en p{\scriptsize a}}^2}}{\en n}$ γύρους είναι $\leq 1-\frac{1}{{\en n}^{\en c-1}}$ ,δηλαδή {\en w.h.p}.\\															
																								\qed
																								
	\pagebreak																								
	\section*{\gr \textbf{Β. Άσκηση 2}}	
	
	\gr \large{Περιγραφή αλγορίθμου:}\\
	
	\gr \normalsize Σύμφωνα με τις διαφάνειες ο γρήγορος αλγόριθμος για χρωματισμό καταφέρνει μετά από $\log_{}^*{{\en n}}$ γύρους να περιορίσει τον αριθμό των χρωμάτων από {\en x} σε 6.Βασιζόμενοι στον παραπάνω αλγόριθμο και στο γεγονός ότι σε κάθε γύρο μειώνει τον αριθμό των χρωμάτων από {\en x} σε Ο($\log_{}{{\en x}}$) τότε μπορούμε να περιγράψουμε τον παρακάτω αλγόριθμο :
	\\
	Έστω ότι έχουμε 6 χρώματα (0,1,2,3,4,5).Άν ένας κόμβος έχει ένα από τα χρώματα 3,4,5 τότε εκτελούνται τα παρακάτω βήματα: Ξαναχρωματίζουμε τον συγκεκριμένο κόμβο με το χρώμα του γονέα του και αντίστοιχα μόλις γίνει αυτό ο γονέας επιλέγει ένα από τα χρώματα {0,1,2}.\\
	Η διαδικασία επιλογής χρώματος του γονέα γίνεται με βάση το χρώμα που έχει ο δικός του γονέας.Οπότε έστω ότι επιλέγει το μικρότερο αποδεκτό χρώμα από τα {0,1,2} ώστε να διατηρηθεί η αρχή πυ λέει πως γειτονικοί κόμβοι δεν μπορούν να έχουν το ίδιο χρώμα.Δηλαδή π.χ άμα ο γονιός του είχε χρώμα 0 τότε αυτός θα χρωματιζόταν με 1.
	\\
	Σύμφωνα με αυτό το {\en loop} συνεχίζουμε σε κάθε γύρο μέχρι να καταλήξουμε σε έναν αποδεκτό χρωματισμό του δένδρου που θα περιλαμβάνει μόνο 3 χρώματα χωρίς την ύπαρξη {\en adjacent} κόμβων.
	
	
		
		
		
		
\end{document}